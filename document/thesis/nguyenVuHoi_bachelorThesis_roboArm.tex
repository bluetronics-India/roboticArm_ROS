\documentclass{article}

\usepackage{amsmath}
\usepackage{graphicx}
\usepackage{tikz}
\usetikzlibrary{calc}
\usetikzlibrary{decorations.pathmorphing}

\pdfinfo { /Title  (Robotic Arm Thesis)
               /Creator (Nguyen Vu Hoi)
               /Author (Nguyen Vu Hoi vuhoinguyen@gmail.com)
               /CreationDate (D:20030101000000)  %format D:YYYYMMDDhhmmss
               /ModDate (D:20030815213532)
               /Subject (Writing a Bachelor thesis about robotic arm)
               /Keywords (Bachelor, Thesis, Robotic, Arm)}
\pdfcatalog { /PageMode (/UseOutlines)
                  /OpenAction (fitbh)  }
                  
\title{\textbf{Thesis Name}}
\date{2015-04-08 (to be changed)}
\author{\underline{Nguyen} Vu Hoi}

\begin{document}
  \pagenumbering{gobble}
  \centering
  \maketitle
\begin{tikzpicture}[overlay,remember picture]
    \draw [line width=1mm,decorate,decoration={
        }]
        ($ (current page.north west) + (1cm,-1cm) $)
        rectangle
        ($ (current page.south east) + (-1cm,1cm) $);
   
\end{tikzpicture}
  
  \newpage
  \tableofcontents
  \listoffigures
  
  \newpage
  \pagenumbering{roman}
  
  \section{Abstract}
  This document describes the sequence to build a robotic arm; including modelling, simulation, real-time control and implementation.
  
  \newpage
  \pagenumbering{arabic}
  
  \section{Introduction}
  Introduction
  
  \newpage
  \section{State of the art}
  State of the art
  
  \newpage
  \subsection{Kinematics}
  
  \newpage
  \subsection{Inverse Kinematics}
  
  \newpage
  \section{Development and analysis}
  Abstract
  
  \newpage
  \section{Implementation}
  Abstract
  
  \newpage
  \section{Conclusion}
  Abstract
  
\end{document}